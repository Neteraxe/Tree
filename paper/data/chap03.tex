% !TeX root = ../Tree.tex

\chapter{数据库中的树}

\section{四叉树,八叉树}

四叉树,八叉树是两种$m$路树的应用,这两种树的$m$均固定,分别为4和8。

八叉树是一种树型数据结构,其中每个内部节点正好有八个子节点。八叉树最常用于通过递归地将三维空间细分为八个八叉树来划分空间。而四叉树其中每个内部节点正好有四个子节点。四叉树是八叉树的二维模拟,通常用于通过递归地将二维空间细分为四个象限或区域来划分二维空间。

\begin{enumerate}
	\item 将空间分解成适应性单元。
	\item 每个单元有一个最大容量。当达到最大容量时,单元分解。
	\item 树目录遵循四叉树的空间分解。
\end{enumerate}

四叉树,八叉树主要用于空间数据的压缩和相关图像处理。

\section{B树,B+树,B*树}

B树\cite{enwiki:1024896978}是一种自平衡树数据结构,它维护已排序的数据,并允许在对数时间内进行搜索、顺序访问、插入和删除。与其他自平衡$m$路搜索树不同,B树非常适合读写相对较大数据块的存储系统,如磁盘。

B树是由Rudolf Bayer和Edward M. McCreight在波音研究实验室工作时,为了高效地管理大量随机访问文件的索引页而发明。B树的索引非常庞大,从而具有大量索引页。B的意思可能代表“波音,平衡,宽广,浓密,拜耳”($Boeing, balanced, broad, bushy, and Bayer$)等多个意思。术语B树可以指特定的设计,也可以指一般的设计类别。在狭义上,B树在其内部节点中存储键,但不需要在叶子处的记录中存储这些键。一般类包括B+树和B*树等变体。

依据Donald Knuth的定义,$m$阶B树是满足以下性质的树:

\begin{enumerate}
	\item 每个结点最多有$m$个子树。
	\item 每个非叶子结点(除了根)有至少 $⌈m/2⌉$ 个子节点。
	\item 如果根结点不是叶子结点,根结点有最少两个子树。
	\item 一个有k个子树的非叶子结点包含$k-1$个键。
	\item 所有的叶子都出现在同一水平面上,不携带任何信息。
\end{enumerate}

每个内部节点(非叶子结点与根节点)的键分隔其子树。例如,如果一个内部节点有3个子树,那么它必须有2个键:$a_1$和$a_2$。最左侧子树中的所有值小于$a_1$,中间子树中的所有值介于$a_1$和$a_2$之间,最右侧子树中的所有值大于$a_2$。

《大型有序索引的组织与维护》中首先对B树进行了描述。没有一篇论文正式介绍过B+树的概念。相反,维护叶节点中所有数据的概念作为一个有趣的变体被反复提出。Douglas Comer对B+树的早期调查也包括B+树。Comer指出,B+树用于IBM的VSAM数据访问软件,他引用了IBM在1973年发表的一篇文章。

一个B+树的数据完全存储在叶子上,而不是有一部分存储在内部节点上的键。B+树将键的副本存储在内部节点中;键和记录存放在叶子结点中;此外,叶节点可以包括指向下一个叶节点的指针,以加速顺序访问,这通常使用链表实现。

B*树平衡更多相邻的内部节点,以保持内部节点更密集。此变体确保非根节点至少占满2/3而不是1/2。因为在B树中插入节点的操作最昂贵的部分是分割节点,创建B*-树是为了尽可能推迟拆分操作。为了保持这一点,与其在节点满时立即拆分节点,不如将其键与旁边的节点共享。这种溢出操作的成本比拆分要低,因为它只需要在现有节点之间移动键,而不需要为新节点分配内存。对于插入,首先检查节点中是否有一些可用空间,如果有,则只需将新键插入节点。但是,如果节点已满(它有$m−1$个键,其中$m$是树作为从一个节点指向子树的指针的最大数目),需要检查是否存在正确的兄弟,以及是否有一些可用空间。如果右同胞有$j<m− 1$个键,然后在两个兄弟节点之间均匀地重新分配键。为此,来自当前节点的$m-1$键、插入的新键、来自父节点的一个键和来自兄弟节点的$j$个键被视为$m+j+1$键的有序数组。数组被分成两半,这样$⌊(m+j+1)/2⌋$最低的键保留在当前节点中,下一个(中间)键插入父节点,其余的键插入到右兄弟节点。(新插入的键可能会在三个位置中的任何一个结束。)右兄弟节点已满,而左兄弟节点未满的情况类似。当两个兄弟节点都已满,然后将两个节点(当前节点和兄弟节点)拆分为三个节点,并在树上向上移动一个键到父节点。如果父节点已满,则进行递归的溢出和拆分操作。

B树是一种$m$路树的应用,满足下列最佳情况$h_{min}高度$和最坏情况$h_{max}$高度:
$$h_{min} = ⌈\log_m(n+1)⌉-1$$
$$h_{max} = ⌈\log_{\frac{m}{2}}(\frac{n+1}{2})⌉$$



